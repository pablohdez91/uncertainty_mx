\documentclass[11pt,a4paper]{article}
\usepackage[utf8]{inputenc}
\usepackage{amsmath}
\usepackage{amsfonts}
\usepackage{amssymb}
\usepackage{titling}

\title{Medición de la Incertidumbre Económica en México: Una Aproximación Basada en Análisis de Texto}
\date{Abril 2024}
\author{Juan Pablo Hernández Reyes}

\begin{document}
\maketitle

\begin{abstract}
Se construyen indicadores de incertidumbre por política económica, mediante el análisis de texto de artículos periodísticos para México. 
Utilizando la metodología propuesta por \cite{azqueta_2017}, la cual se basa en la medición de la cobertura relativa de los artículos periodísticos, se crean cuatro indicadores temáticos (Política Monetaria, Política Fiscal, Política Comercial y Riesgo Político) y uno de incertidumbre por Política Económica en general, para el periodo enero 1993 - mayo 2020. 
Esta metodología se fundamenta en un algoritmo de clasificación no supervisado conocido como \textit{Latent Dirichlet Allocation} (LDA). \\

Se encuentra que los picos de incertidumbre son consistentes con eventos históricos clave, como la crisis económica de 1994-1995, la gran recesión de 2008 o la reciente crisis por COVID-19. 
El pico de incertidumbre más alto ocurrió por la crisis económica de 1994-1995. Mientras que el más persistente ha sido el actual, afectado entre otras cosas por la pandemia de COVID-19. El pico de incertidumbre ocasionado por la pandemia, comienza en enero de 2020 y ha tenido su punto más alto el mes de marzo de este mismo año. 
Respecto a la relación de la incertidumbre con variables macroeconómicas. Se observan efectos negativos y persistentes en el cíclo económico, negativos de corto plazo en la inversión y positivos en la tasa de interés. 
También, se encuentra que la incertidumbre por política fiscal tiene efectos con una magnitud ligeramente mayor sobre el cíclo económico y la inversión, respecto a los otros indicadores de incertidumbre.  Por otro lado, el comportamiento de los mercados financieros es el único factor que tiene efectos significativos (negativos) sobre la incertidumbre.
\end{abstract}

\section{Introducción}
¿Qué se hace? \\
¿Por qué se hace? \\
¿Cómo se hace?

\section{Revisión de la Literatura}
¿Qué se ha hecho en cuanto a medición de la incertidumbre? ¿Cómo se mide la incertidumbre normalmente? \\
¿Qué investigaciones similares se han hecho en otros países?
¿Qué investigaciones similares se han hecho en México? \\
¿Cuáles son las causas y los efectos que tiene la incertidumbre?

\section{Construcción de los Índices de Incertidumbre Económica}

\section{Análisis de la Inceridumbre Económica}

\end{document}